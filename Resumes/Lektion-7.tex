\documentclass[a4wide,10pt]{article}
%\usepackage{a4wide}
\usepackage[applemac,utf8]{inputenc}
\usepackage[danish]{babel}
\usepackage[T1]{fontenc}
\usepackage{pdfsync}
\usepackage{amsmath,amssymb,amsfonts} 
\usepackage[pdftex]{graphicx}
\usepackage{wrapfig}
\usepackage{color}
\usepackage[small,bf]{caption}

\begin{document}
\title{COS3 lektion 7}
\author{Nis Sarup}
\date{\today}
\maketitle


\addtocounter{section}{6}
\section{Deadlocks} % (fold)
\label{sec:deadlocks}
\begin{itemize}
	\item Deadlock: When two processes are waiting for each other and neither can do anything before the other.
\end{itemize}
\subsection{System Model} % (fold)
\label{sub:system_model}
\begin{itemize}
	\item System Model: Each resource must be REQUESTED before USE and RELEASED again.
	\item Deadlock if two processes holds a resource each and will not RELEASE it until it has gotten access to the one the other process holds.
\end{itemize}
% subsection system_model (end)

\subsection{Deadlock Characterization} % (fold)
\label{sub:deadlock_characterization}
\begin{itemize}
	\item Deadlock: Processes never finish executing and system resources are tied up, preventing other jobs from starting.
	\item Necessary conditions:
	\begin{itemize}
		\item Mutual exclusion:At least one resource must be held in a non-shareable mode.
		\item Hold and Wait: A process must be holding at least one resource and be waiting to acquire additional resources currently held by other processes.
		\item No preemption: Resources can only be released by the process holding it.
		\item Circular wait: P1 waits for P2 which waits for P3 which waits for p4... 
	\end{itemize}
\end{itemize}
% subsection deadlock_characterization (end)

\subsection{Methods for Handling Deadlocks} % (fold)
\label{sub:methods_for_handling_deadlocks}
\begin{itemize}
	\item A protocol to prevent and avoid deadlocks.
	\item Enable the system to detect and recover from deadlocks.
	\item Ignore and pretend deadlocks never happens.
\end{itemize}
% subsection methods_for_handling_deadlocks (end)

\subsection{Deadlock Prevention} % (fold)
\label{sub:deadlock_prevention}
\begin{itemize}
	\item Ensure that at least one fo the four conditions for a deadlock cannot hold.
\end{itemize}
% subsection deadlock_prevention (end)

\subsection{Deadlock Avoidance} % (fold)
\label{sub:deadlock_avoidance}
\begin{itemize}
	\item The system can identify which REQUESTs for resources would result in a deadlock and makes the requesting process wait until the REQUEST would no longer result in a deadlock.
\end{itemize}
% subsection deadlock_avoidance (end)

\subsection{Deadlock Detection} % (fold)
\label{sub:deadlock_detection}
\begin{itemize}
	\item Detection of deadlocks requires two thing:
	\begin{itemize}
		\item A deadlock detecting algorithm.
		\item An algorithm to recover from the deadlock.
	\end{itemize}
	\item It is necessary to evaluate how often to run the detecting algorithm.
	\item Once every resource REQUESt would be too often.
	\item When CPU utilization drops below 40\% is more appropriate.
\end{itemize}
% subsection deadlock_detection (end)

\subsection{Recovery from Deadlocks} % (fold)
\label{sub:recovery_from_deadlocks}
\begin{itemize}
	\item Process Termination:
	\begin{itemize}
		\item Abort all deadlocked processes.
		\begin{itemize}
			\item Expensive
			\item Much will need to be recomputed.
		\end{itemize}
		\item Abort one deadlocked process until deadlock resolves.
		\begin{itemize}
			\item Deadlock detecting algorithm must run after each termination.
			\item System must decide which process to terminate from some parameters: priority, runtime, etc.
		\end{itemize}
	\end{itemize}
	\item Resource preempting has issues:
	\begin{itemize}
		\item Which resource to preempt? Which parameters to consider.
		\item Should the resource be rolled back?
		\item How to prevent starvation (The process is always going to be preempted on the same resource).
	\end{itemize}
\end{itemize}
% subsection recovery_from_deadlocks (end)
% section deadlocks (end)
\end{document}